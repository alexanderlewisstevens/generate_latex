% Example problem 108
\question Beans are a type of legume that are packed with nutrients, making them a valuable source of protein in our diets. However, beans are more than just a nutritious food source – they also play an important role in sustainable agriculture. Beans have the ability to fix nitrogen from the air into the soil, which is a vital nutrient for plant growth. This makes them a valuable crop in promoting sustainable farming practices and improving soil health.

Nitrogen is one of the essential nutrients for plant growth and is a key component of fertilizers. Traditional farming practices involve the use of synthetic fertilizers, which are expensive and can have negative impacts on the environment. The production of synthetic fertilizers requires a large amount of energy, primarily derived from fossil fuels, contributing to greenhouse gas emissions and climate change. Moreover, the overuse of synthetic fertilizers can lead to nutrient runoff, polluting waterways and causing harmful algal blooms.

Beans, on the other hand, have the ability to fix nitrogen from the air through a symbiotic relationship with bacteria called rhizobia. These bacteria live in nodules on the roots of legumes and convert atmospheric nitrogen into a form that plants can absorb and use. This process, known as nitrogen fixation, not only provides plants with the essential nutrient but also enriches the soil with organic matter, improving its overall health and fertility.

By incorporating beans into crop rotations, farmers can reduce their reliance on synthetic fertilizers, leading to cost savings and a decrease in the negative environmental impacts associated with their production and use. Research has shown that including beans in a crop rotation can decrease synthetic fertilizer use by up to 50%, resulting in a substantial reduction in greenhouse gas emissions.

Furthermore, beans contribute to sustainable farming practices by promoting biodiversity and soil health. Legumes, including beans, add organic matter to the soil, improving its structure and water-holding capacity. This makes the soil less susceptible to erosion and increases its ability to support plant growth. The increased levels of organic matter also stimulate the growth of beneficial microorganisms, which play a vital role in nutrient cycling and plant health.

Moreover, bean crops are generally low-input and can be grown without the use of synthetic pesticides, making them a more environmentally friendly option. This is because beans are considered a self-pollinating crop, meaning they do not require pollinators or cross-pollination, reducing the risk of exposure to harmful chemicals for pollinators and other beneficial insects.

In addition to their benefits for soil health and sustainable farming
\begin{solution}
<solution here>
\end{solution}
